\documentclass[a4page]{article}
\usepackage[11pt]{extsizes}
\usepackage[T2A]{fontenc}
% \usepackage[utf8]{inputenc}
\usepackage[russian]{babel}
\usepackage{amsmath}
% \usepackage[left=20mm, top=15mm, right=15mm, bottom=30mm, footskip=15mm]{geometry}
\usepackage{indentfirst}
\usepackage{paralist}

\usepackage{fancyvrb}
\usepackage{framed}
\usepackage{url}
\usepackage{csquotes}
\usepackage{datetime2}

\usepackage{graphicx}
\usepackage{svg}
\graphicspath{ {./images/} }

% \usepackage{float}
% \floatstyle{ruled}


\author{Григорий Матюхин \and Даниил Генералов}
\date{\DTMDisplaydate{2024}{2}{27}{-1}}
\title{Техническое задание\\ \Large<<Система предоставления и запроса вакансий для бюро по трудоустройству>>}

\begin{document}
\maketitle
\thispagestyle{empty}

\newpage

\tableofcontents

\newpage

\section{Введение}

\subsection{Наименование программы}

Наименование программы - <<Работы.нет! и не будет!>>.

\subsection{Краткая характеристика области применения}

Система <<Работы.нет! и не будет!>> предназначена для организации сбора, индексации и визуального представления открытых вакансий различных работодателей, выставляющих свои вакансии в бюро по трудоустройству <<Работы.нет!>>. Работодатель должны иметь возможность размещать информацию об открытых вакансиях, а соискатели отвечать на доступные вакансии.

\section{Основания для разработки}

Основанием для разработки является договор 42 от 11.11.2024. Договор утвержден директором ООО <<Работы.нет!>> Христодула Акакием Евлампиевичем, именуемым в дальнейшем Заказчиком, и Добрыйвечер Александром Михайловичем (самозанятый), именуемым в дальнейшем Исполнителем.

Согласно Договору, Исполнитель обязан разработать и установить систему <<Работы.нет! и не будет!>> на оборудовании Заказчика не позднее 20.12.2024, предоставить исходные коды и документацию к разработанной системе не позднее 20.12.2024.

Наименование темы разработки – <<Разработка системы предоставления и запроса вакансий для бюро по трудоустройству Работы нет! и не будет!>>.
Условное обозначение темы разработки (шифр темы) – <<Работы-нет-42>>.

\section{Назначение разработки}

Программа будет использоваться двумя группами пользователей: работодатель и соискатель.

\subsection{Функциональное назначение}

Для работодателя программа предоставляет возможность размещать, изменять, отзывать, отмечать тэгами и закрывать вакансии, а также редактировать информацию в профиле компании.

Для соискателя программа предоставляет позможность просматривать отрытые вакансии, фильтровать их по тэгам, откликаться на интересующие вакансии, а также изменять информацию в своем профиле.

Как работодатель, так и соискатель могут в любое время удалить свой профиль.

\subsection{Эксплуатационное назначение}

Программа должна эксплуатироваться в виде веб страницы на домене Заказчика.

\section{Требования к программе или пограммному изделию}

\subsection{Требования к функциональным характеристикам}

\subsubsection{Требования к составу выполняемых функций}

После навигации на домен Заказчика пользователю отображается форма ввода логина и пароля.

Пользователь также имеет возможность создать новый аккаунт. В таком случае ему предлагается ввести имя пользователя, пароль и установить тип аккаунта (работодатель или соискатель). После чего пользователь опять переходит на страницу с формой ввода логина и пароля.

После введения логина и пароля веб-интерфейс посылает запрос серверу, который проверяет имя пользователя и пароль и открывает интерфейс, соответствующи типу пользователя (работодатель или соискатель).

Для работодателя веб-интерфейс предоставляет следующие возможности:

\begin{itemize}
  \item просмотр опубликовынных данным работодателем вакансий;
  \item создавать новые вакансии;
  \item редактирование отдельной вакансии;
  \item отзыв (удаление) одной или нескольких вакансий;
  \item закрытия (отметка недоступной для отклика) вакансии;
  \item просмотр откликнувшихся на вакансию соискателей;
  \item просмотр профиля отдельного откликнувшегося соискателя;
  \item редактирование своего компании;
  \item удаление своего профиля.
\end{itemize}

При просмотре опубликованных вакансий, работодатель видит список с их названиями.

При создании вакансии, открывается графический интерфейс текстового редактора. При нажатии кнопки <<Сохранить>>, вакансия публикуется и добавляется в вышеназванный список. Работодатель также может добавить тэги к вакансии перед публикацией.

При редактировании вакансии открывается такой-же интерфейс, как и при создании новой вакнсии. Работодатель может изменять содержимое вакансии на свое усмотрение.

Работодатель может выбрать одну или несколько вакансий и удалить их. После чего они пропадают из списка вакансий.

При закрытии вакансии, она не удаляется их списка, но на нее больше нельзя откликнутся.

При просмотре откликнувшихся на вакансию соискателей, работодатель видит список с именами людей.

При выборе кого-либо их списка откликнувшихся, работодатель видит полный профиль соискателя.

Работодатель может редактировать информацию доступную в своем профиле, при помощи графического интерфеся текстового редактора.

Работодатель может удалить свой аккаунт. В таком случае все данные связанные с ним удаляются из системы.

Для соискателя веб-интерфейс представляет следующие возможности:

\begin{itemize}
  \item просмотр вакансий, на которые соискатель уже откликнулся;
  \item поиск вакансий по названию вакансии или работодателя;
  \item фильтрация отабраженных вакансий по тэгам;
  \item отклик на вакансию;
  \item просмотр профиля работодателя;
  \item редактирование своего профиля;
  \item удаление своего профиля.
\end{itemize}

Основной интерфейс соискателя предоставляет из себя интерфейс поисковой системы. Соискатель может ввести имя вакансии или работодателя. Вакансии, удовлетворяющие поисковому запросу, можно фильтровать используя проставленные работодателем тэги.

Соискатель может откликнуться на вакансию. В таком случае эта вакансия добавляется в список, на которые соискатель уже откликнулся, а сам соискатель --- в список откликнувшихся на вакансию в интерфесе работодателя.

Соискатель может посмотреть полный профиль работодателя, кликнув на него.

Редактирование и удаление профиля пользователя функционирует так же как редактирование и удаление профиля работодателя.

\subsubsection{Требования к организации входных и выходных данных}

Все данные о пользователях хранятся в базе данных. СУБД обеспечивает разграничение прав доступа к данным --- дает работодателю возможность только читать информацию о соискателе, но не редактировать ее, а соискателю --- читать информацию о вакансиях и работодателе, но не редактировать их.

\subsubsection{Требования к временным характеристикам}

После изменения данных работодателя или соискателя, находящихся в базе данных, новая информация должна отображается не позднее, чем через 5 секунд.

\subsection{Требования к надежности}

Вероятность безотказной работы системы должна составлять не менее 99.99\% при условии исправности сети.

\subsubsection{Требования к обеспечению надежного функционирования программы}

В связи с тем, что в базе данных хранятся личные данные соискателей --- база данных должна быть зашифрована, и вся информация в ней должна быть резервирована.

Надежное (устойчивое) функционирование программы должно быть обеспечено выполнением Заказчиком совокупности организационно-технических мероприятий, перечень которых приведен ниже:

\begin{itemize}
  \item организацией бесперебойного питания технических средств;
  \item использованием лицензионного программного обеспечения;
  \item регулярным выполнением рекомендаций Министерства труда и социального развития РФ, изложенных в Постановлении от 23 июля 1998 г. <<Об утверждении межотраслевых типовых норм времени на работы по сервисному обслуживанию ПЭВМ и оргтехники и сопровождению программных средств>>;
  \item регулярным выполнением требований ГОСТ 51188-98. Защита информации. Испытания программных средств на наличие компьютерных вирусов.
\end{itemize}

\subsubsection{Время восстановления после отказа}

Время восстановления после отказа, вызванного сбоем электропитания технических средств (иными внешними факторами), а не фатальным сбоем операционной системы, не должно превышать 10 минут при условии соблюдения условий эксплуатации технических и программных средств.

Время восстановления после отказа, вызванного неисправностью технических средств, фатальным сбоем (крахом) операционной системы, не должно превышать времени, требуемого на устранение неисправностей технических средств и переустановки программных средств.

\subsubsection{Отказы из-за некорректных действий оператора}

Отказы программы не должны быть возможны вследствие некорректных действий оператора (пользователя). 

\subsection{Условия эксплуатации}

Программа-сервер запускается на одном или нескольких компьютерах или виртульных машинах Заказчика. База даных находится на отдельном компьютере или виртуальной машине. Дожна существовать устоичивая связь по сети между сервером (серверами) и базой данных.

Пользователи получеют доступ по сети Интернет, используя свои устройства. Сервер (серверы) должны выдерживать большое количество одновременых клиентов.

\subsubsection{Требования к численности и квалификации персонала}

При установке и настройке системы необходим системный администратор.

Системный администратор должен иметь высшее профильное образование и сертификаты компании-производителя операционной системы. В перечень задач, выполняемых системным администратором, должны входить:

\begin{itemize}
  \item настройка сервера (серверов);
  \item настройка СУБД;
  \item настройка сети между сервером (серверами) и СУБД.
\end{itemize}

К квалификации посетителя веб-страницы специальные требования не предъявляются.

\subsection{Требования к составу и параметрам технических средств}

Минимальный состав технических средств: 

\begin{itemize}
  \item Один или два компьютера (или виртуальные машины) для сервера, включающие в себя:
    \begin{itemize}
      \item процессор x86 с тактовой частотой, не менее 1 ГГц;
      \item оперативную память объемом, не менее 1 Гб;
      \item видеокарту, монитор, клавиатуру.
    \end{itemize}
    Компьютер (компьютеры) для сервера должны иметь доступ в сеть Интернет.
  \item Два компьютера (или виртуальные машины) для СУБД (основной и резервный), включающие в себя: 
    \begin{itemize}
      \item процессор x86 с тактовой частотой, не менее 1 ГГц;
      \item оперативную память объемом, не менее 1 Гб;
      \item видеокарту, монитор, клавиатуру.
    \end{itemize}
\end{itemize}

\subsection{Требования к информационной и программной совместимости}

Сервер (серверы) обмениваются с СУБД сообщениями по локальной сети, при этом используется протокол HTTP. Должно быть исключено появление посторонних устройств в сети.

Для разработки программы должны использоваться следующие языки программирования высокого уровня и разметки данных:

\begin{itemize}
  \item Rust;
  \item HTML5;
  \item CSS;
  \item JavaScript;
  \item WebAssembly.
\end{itemize}

Допускается использование вспомогательных библиотек ПО с открытым исходным кодом.

Разрабатываемое ПО не должно основываться на программных продуктах, требующих лицензионных отчислений от пользователей.

Разрабатываемое ПО должно функционировать в операционных системах семейства Linux.

Разрабатываемое ПО должно обеспечивать доступ пользователей к своей функциональности посредством HTML-браузеров.

\subsection{Специальные требования}

Программа должна обеспечивать взаимодействие с пользователем посредством графического пользовательского веб-интерфейса, разработанного согласно рекомендациям компании-производителя операционной системы.

\section{Требования к программной документации}

Предварительный состав программной документации:

\begin{itemize}
  \item техническое задание (включает описание применения);
  \item программа и методика испытаний;
  \item руководство системного программиста;
  \item руководство оператора;
  \item руководство программиста;
  \item ведомость эксплуатационных документов;
  \item формуляр.
\end{itemize}

\section{Технико-экономические показатели}

Программа <<Работы.нет! и не будет!>> пригодна для развертывания в условиях высокой доступности. Программа будет использоваться работодателями и соискателями по всей стране. В связи с ростом экономики и населения, вакансий и соискателей становится все больше и больше --- следует ожидать большого роста годовой потребности.

\section{Стадии и этапы разработки}

Разработка должна быть проведена в три стадии:

\begin{enumerate}
  \item техническое задание;
  \item технический (и рабочий) проекты;
  \item внедрение.
\end{enumerate}

На стадии <<Техническое задание>> должен быть выполнен этап разработки, согласования и утверждения настоящего технического задания.

На стадии <<Технический (и рабочий) проект>> должны быть выполнены перечисленные ниже этапы работ:

\begin{itemize}
  \item разработка программы;
  \item разработка программной документации;
  \item испытания программы.
\end{itemize}

На стадии <<Внедрение>> должен быть выполнен этап разработки <<Подготовка и передача программы>>.

Содержание работ по этапам:
На этапе разработки технического задания должны быть выполнены перечисленные ниже работы:

\begin{itemize}
  \item постановка задачи;
  \item определение и уточнение требований к техническим средствам;
  \item определение требований к программе;
  \item определение стадий, этапов и сроков разработки программы и документации на нее;
  \item согласование и утверждение технического задания.
\end{itemize}

На этапе разработки программы должна быть выполнена работа по программированию (кодированию) и отладке программы.

На этапе разработки программной документации должна быть выполнена разработка программных документов в соответствии с требованиями ГОСТ 19.101-77.

На этапе испытаний программы должны быть выполнены перечисленные ниже виды работ:

\begin{itemize}
  \item разработка, согласование и утверждение порядка и методики испытаний;
  \item проведение приемо-сдаточных испытаний;
  \item корректировка программы и программной документации по результатам испытаний.
\end{itemize}

На этапе подготовки и передачи программы должна быть выполнена работа по подготовке и передаче программы и программной документации в эксплуатацию на объектах заказчика.

\section{Порядок контроля и приемки}

Приемосдаточные испытания программы должны проводиться согласно разработанной исполнителем и согласованной заказчиком <<Программы и методики испытаний>>.

Ход проведения приемосдаточных испытаний заказчик и исполнитель документируют в протоколе испытаний.
На основании протокола испытаний исполнитель совместно с заказчиком подписывают акт приемки-сдачи программы в эксплуатацию. 

\end{document}
