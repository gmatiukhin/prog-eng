\documentclass[a4page]{article}
\usepackage[14pt]{extsizes} % для того чтобы задать нестандартный 14-ый размер шрифта
\usepackage[utf8]{inputenc}
\usepackage[russian]{babel} % поддержка русского языка
\usepackage{amsmath}  %  математические символы
\usepackage[left=20mm, top=15mm, right=15mm, bottom=30mm, footskip=15mm]{geometry} % настройки полей документа
\usepackage{indentfirst} % по умолчанию убирается отступ у первого абзаца в секции, это отменяет это.
\usepackage{paralist}

\usepackage{minted}
\usepackage{fancyvrb}
\usepackage{framed}
\usepackage{url}
\usepackage{csquotes}
\usepackage{datetime2}

\usepackage{graphicx}
\usepackage{svg}
\graphicspath{ {./diagrams/output} }

% \usepackage{float}
% \floatstyle{ruled}

%\renewcommand{\baselinestretch}{1.35}


\begin{document} % начало документа

% НАЧАЛО ТИТУЛЬНОГО ЛИСТА
\begin{titlepage}

\begin{center}
\hfill \break
\textbf{
\large{РОССИЙСКИЙ УНИВЕРСИТЕТ ДРУЖБЫ НАРОДОВ}\\
\normalsize{Факультет физико-математических и естественных наук}\\ 
\normalsize{Кафедра математического моделирования и искуственного интеллекта}\\
}
\vspace*{\fill}
\Large{\textbf{Техническое задание\\ \Large<<Система предоставления и запроса вакансий для бюро по трудоустройству>>}}
\vspace*{\fill}

\end{center}
 
 \begin{flushright}
 Студент: \underline{Матюхин Григорий и Генералов Даниил}\\ \vspace{0.5cm}
 Группа: \underline{НПИбд-01-21}
 \end{flushright}
 
 
\begin{center} \textbf{МОСКВА} \\ 2024 г. \end{center}
\thispagestyle{empty} % выключаем отображение номера для этой страницы

\end{titlepage}
 % КОНЕЦ ТИТУЛЬНОГО ЛИСТА

\newpage

\tableofcontents

\newpage

\section{Введение}

Система <<Работы.нет! и не будет!>> предназначена для организации сбора, индексации и визуального представления открытых вакансий различных работодателей, выставляющих свои вакансии в бюро по трудоустройству <<Работы.нет!>>. Работодатель должны иметь возможность размещать информацию об открытых вакансиях, а соискатели отвечать на доступные вакансии.
Целью работы является разработка, отладка и разворачивание системы на объектах заказчика.

Диаграммы бизнес-процессов и use-case сделаны с помощью PlantUML.

\newpage
\section{Постановка задачи}

Программа <<Работы.нет! и не будет!>> должна предоставлять функции для работы с вакансиями в бюро по трудоустройству <<Работы.нет!>>.

Программа имеет два типа пользователей: работодатели и соискатели. Работодатели имеют возможность создавать вакансии,
а соискатели имеют возможность отвечать на существующие вакансии.

Оба типа пользователей имеют профиль, который содержит имя, фотографию/логотип, текстовое поле для общего описания и прикрепляемые файлы.
Вакансии также имеют профиль. Профили имеют дополнительную информацию в зависимости от того, какой тип сущности связан с ними:
например, соискатель имеет информацию об образовании/опыте работы, работодатель -- адрес и ссылку на веб-сайт,
а вакансия -- зарплату и тэги.

Для работодателя программа предоставляет возможность размещать, изменять, отзывать, отмечать тэгами и закрывать вакансии, а также редактировать информацию в профиле компании.

Для соискателя программа предоставляет позможность просматривать открытые вакансии, фильтровать их по тэгам, откликаться на интересующие вакансии, а также изменять информацию в своем профиле.

\newpage
\section{Бизнес-процессы}

Основной бизнес-процесс в системе -- это взаимодействие работодателей с соискателями.
Сначала работодатель создает вакансию, и она появляется в поисковой выдаче.
Затем соискатель выполняет поиск, пока не найдет интересующую его вакансию.
После этого он откликается на вакансию,
что будет видно в личном кабинете работодателя.
Затем работодатель связывается с соискателем (вне контекста данной системы)
и, возможно, закрывает вакансию.

\includegraphics[width=300pt]{bus-proc.plantuml-rendered.png}

\newpage
\section{Use-case диаграмма}

Работодатель и соискатель должны иметь доступ к списку вакансий.
Работодатель создает, редактирует, закрывает и удаляет вакансии от имени
своего аккаунта.
Работодатель также может просмотреть информацию о вакансии.

Соискатель тоже видит информацию о вакансии, в том числе в поисковой выдаче.
Соискатель имеет профиль, который он должен иметь возможность редактировать.
При выборе вакансии соискатель может откликнуться на нее:
это добавляет его профиль в список профилей, которые видит работодатель для данной вакансии.

\includegraphics[width=190pt]{use-case.plantuml-rendered.png}

\newpage
\section{Заключение}

Разработка системы <<Работы.нет! и не будет!>> является важным стратегическим приоритетом компании <<Работы.нет!>>,
позволяя ей выйти на межрегиональный и международный рынок объявлений.
Учитывая это, при разработке системы следует использовать архитектуру, которая способна поддерживать высокий уровень нагрузки
и повышенную степень защиты информации.

Основанием для разработки является договор 42 от 11.11.2024.
Согласно договору, Исполнитель обязан разработать и установить систему <<Работы.нет! и не будет!>> на оборудовании Заказчика не позднее 20.12.2024.

\newpage
\section{Приложение: исходный код диаграм}

Рисунки сделаны с помощью PlantUML: это программа для создания схем и диаграмм
на основании текстового исходного кода.

\subsection{PlantUML для диаграммы бизнес-процессов}

\inputminted[fontsize=\footnotesize]{text}{diagrams/bus-proc.plantuml}

\subsection{PlantUML для use-case диаграммы}

\inputminted[fontsize=\footnotesize]{text}{diagrams/use-case.plantuml}

\end{document}
