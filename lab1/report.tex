\documentclass[a4page]{article}
\usepackage[11pt]{extsizes}
\usepackage[T2A]{fontenc}
% \usepackage[utf8]{inputenc}
\usepackage[russian]{babel}
\usepackage{amsmath}
% \usepackage[left=20mm, top=15mm, right=15mm, bottom=30mm, footskip=15mm]{geometry}
\usepackage{indentfirst}
\usepackage{paralist}

\usepackage{fancyvrb}
\usepackage{framed}
\usepackage{url}
\usepackage{csquotes}
\usepackage{datetime2}

\usepackage{graphicx}
\usepackage{svg}
\graphicspath{ {./diagrams/output} }

% \usepackage{float}
% \floatstyle{ruled}


\author{Григорий Матюхин \and Даниил Генералов}
\date{\DTMDisplaydate{2024}{2}{27}{-1}}
\title{Техническое задание\\ \Large<<Система предоставления и запроса вакансий для бюро по трудоустройству>>}

\begin{document}
\maketitle
\thispagestyle{empty}

\newpage

\tableofcontents

\newpage

\section{Введение}

Система <<Работы.нет! и не будет!>> предназначена для организации сбора, индексации и визуального представления открытых вакансий различных работодателей, выставляющих свои вакансии в бюро по трудоустройству <<Работы.нет!>>. Работодатель должны иметь возможность размещать информацию об открытых вакансиях, а соискатели отвечать на доступные вакансии.
Целью работы является разработка, отладка и разворачивание системы на объектах заказчика.

Диаграммы бизнес-процессов и use-case сделаны с помощью PlantUML.

\newpage
\section{Постановка задачи}

Программа <<Работы.нет! и не будет!>> должна предоставлять функции для работы с вакансиями в бюро по трудоустройству <<Работы.нет!>>.

Программа имеет два типа пользователей: работодатели и соискатели. Работодатели имеют возможность создавать вакансии,
а соискатели имеют возможность отвечать на существующие вакансии.

Оба типа пользователей имеют профиль, который содержит имя, фотографию/логотип, текстовое поле для общего описания и прикрепляемые файлы.
Вакансии также имеют профиль. Профили имеют дополнительную информацию в зависимости от того, какой тип сущности связан с ними:
например, соискатель имеет информацию об образовании/опыте работы, работодатель -- адрес и ссылку на веб-сайт,
а вакансия -- зарплату и тэги.

Для работодателя программа предоставляет возможность размещать, изменять, отзывать, отмечать тэгами и закрывать вакансии, а также редактировать информацию в профиле компании.

Для соискателя программа предоставляет позможность просматривать открытые вакансии, фильтровать их по тэгам, откликаться на интересующие вакансии, а также изменять информацию в своем профиле.

\newpage
\section{Бизнес процессы}

\includegraphics{bus-proc.plantuml-rendered.png}

\newpage
\section{use-case диаграммы}


\includegraphics[width=220pt]{use-case.plantuml-rendered.png}

\newpage
\section{Заключение}

Разработка системы <<Работы.нет! и не будет!>> является важным стратегическим приоритетом компании <<Работы.нет!>>,
позволяя ей выйти на межрегиональный и международный рынок объявлений. 
Учитывая это, при разработке системы следует использовать архитектуру, которая способна поддерживать высокий уровень нагрузки
и повышенную степень защиты информации.

Основанием для разработки является договор 42 от 11.11.2024. 
Согласно договору, Исполнитель обязан разработать и установить систему <<Работы.нет! и не будет!>> на оборудовании Заказчика не позднее 20.12.2024,
предоставить исходные коды и документацию к разработанной системе не позднее 20.12.2024.


\end{document}
